\documentclass[12pt,twoside]{article}

\usepackage{hyperref}
\usepackage{amsmath}
\usepackage{amsfonts}
\newcommand{\profs}{Professors Srini Devadas and Erik Demaine}
\newcommand{\subj}{6.006}

\newlength{\toppush}
\setlength{\toppush}{2\headheight}
\addtolength{\toppush}{\headsep}

\newcommand{\htitle}[3]{\noindent\vspace*{-\toppush}\newline\parbox{6.5in}
{\textit{Introduction to Algorithms: 6.006}\hfill\newline
Massachusetts Institute of Technology \hfill #3\newline
\profs\hfill Handout #1\vspace*{-.5ex}\newline
\mbox{}\hrulefill\mbox{}}\vspace*{1ex}\mbox{}\newline
\begin{center}{\Large\bf #2}\end{center}}

\newcommand{\handout}[3]{\thispagestyle{empty}
 \markboth{Handout #1: #2}{Handout #1: #2}
 \pagestyle{myheadings}\htitle{#1}{#2}{#3}}

\setlength{\oddsidemargin}{0pt}
\setlength{\evensidemargin}{0pt}
\setlength{\textwidth}{6.5in}
\setlength{\topmargin}{0in}
\setlength{\textheight}{8.5in}

\begin{document}


\handout{12}{Problem Set 6}{April 24, 2008}
\setlength{\parindent}{0pt}

\newcommand{\solution}{
  \medskip
  {\bf Solution:}
}

This problem set is due {\bf Thursday May 8} at {\bf 11:59PM}. 

Solutions should be turned in through the course website in PDF form
using \LaTeX\ or scanned handwritten solutions. 

A template for writing up solutions in \LaTeX\ is available on the
course website.

Remember, your goal is to communicate. Full credit will be given only
to the correct solution which is described clearly. Convoluted and
obtuse descriptions might receive low marks, even when they are
correct. Also, aim for concise solutions, as it will save you time
spent on write-ups, and also help you conceptualize the key idea of
the problem.

\medskip

\hrulefill

\medskip

%% Exercises are for extra practice and should not be turned in.

%% {\bf Exercises:}

%% \begin{itemize}

%% %% \item CLRS 24.1-1 (page 591)

%% %% \item CLRS 24.3-2 (page 600)

%% %% \item CLRS 24.3-4 (page 600)

%% %% \item CLRS 24.5-8 (page 614)

%% %% \item CLRS 24.3-6 (page 600)

%% \end{itemize}

%% \hrulefill

\begin{enumerate}

\item {\bf (10 points)} Fibonacci
  
  We define the Fibonacci numbers as follows:
  \begin{align*}
    F_0 &= 0\\
    F_1 &= 1\\
    \forall n > 1, F_n &= F_{n-1} + F_{n-2}
  \end{align*}

  For this problem, you will code four versions of \verb|fib(n)|, all
  of which should return the same answers.

  Download \verb|ps6_fib.zip|.

  \begin{enumerate}
  \item Write \verb|fib_recursive(n)|. It should implement the
    recursion directly and take time exponential in~$n$.
  \item Write \verb|fib_memoize(n)|. It should be recursive like
    \verb|fib_recursive(n)|, but it should memoize its answers, so
    that it runs in time $O(n)$.
  \item Write \verb|fib_bottom_up(n)|. Instead of working top-down
    like in the previous two examples, start from the bottom,
    recording your results in a list. This code should also take
    $O(n)$ time.
  \item Write \verb|fib_in_place(n)|.  It should work bottom-up like
    the previous example, but use only $O(1)$ space instead of
    accumulating answers into a list.
  \end{enumerate}

  Submit \verb|fib.py| to the class website.

\newpage

\item {\bf (15 points)} Making Change

  You've just signed up as a software engineer for a new intergalactic
  trading post in Deep Space 6.  For each transaction, you are given a
  list of coin denomination values,
  e.g., $\mathit{denomination} = [1, 5, 10, 17]$, and an
  amount of change, $C$. You have an unlimited number of each type of
  coin. Your goal is to find the shortest possible list of coins that
  adds up to~$C$. For simplicity, assume that there is always a penny
  $1 \in \mathit{denomination}$ and that the desired change $C$ is an integer,
  so the problem always has a solution.

  \begin{enumerate}
  \item Clearly state the set of subproblems that you will use to
    solve this problem.
  \item Write a recurrence relating the solution of a general
    subproblem to solutions of smaller subproblems.
  \item Analyze the running time of your algorithm, including the
    number of subproblems and the time spent per subproblem. 

    You should end up with a \emph{pseudopolynomial} running time,
    meaning that the polynomial includes some power of~$C$. This is
    not exactly the same as being polynomial with respect to the size
    of the input, because it only takes $\lg C$ bits of input to
    represent the number $C$.
  \item Download \verb|ps6_change.zip|.

    Write a function \verb|make_change(denomination, C)| which returns
    a list of coins that add up to $C$, where the size of the list is
    as small as possible. Write it in a bottom-up manner (because the
    Python recursion stack is limited).

    Note that, assuming your subproblems from part (a) only find the
    size of the best result, you should also keep parent pointers so
    that you can reconstruct the actual subsequence.
    
    Submit \verb|change.py| to the class website.

  \end{enumerate}

\newpage

\item {\bf (15 points)} Making Progress

  You work on your thesis over the weekend, and every time you make a
  change to your code, you run your \verb|test_awesomeness.py| script,
  which spits out a score telling you how awesome your code is. During
  two hard days of work, you accumulate a large, time-ordered list of
  these awesomeness scores, e.g., $[32, 31, 46, 36, 32, 36, 30, 33,
  22, 38, 2, 13]$.

  You have a weekly meeting with your advisor, and each week you have
  to show that you made progress, so that he'll leave you alone for
  another week. You devise a plan in which every week you will show
  your advisor a newer version of your code, along with an awesomeness
  score that is better than the previous week's.  To maximize the
  number of weeks of slacking you get out of your two days of work,
  you need to calculate a longest increasing subsequence of your
  awesomeness scores. In the example, one such subsequence would be
  $[31, 32, 36, 38]$. The subsequence should be strictly increasing,
  because you need to show improvement each time. Thinking back to
  your 6.006 days, you have a vague recollection that longest
  increasing subsequence is one of those problems that can be solved
  by Dynamic Programming.

  \begin{enumerate}
  \item Clearly state the set of subproblems that you will use to
    solve this problem.
  \item Write a recurrence relating the solution of a general
    subproblem to solutions of smaller subproblems.
  \item Analyze the running time of your algorithm, including the
    number of subproblems and the time spent per subproblem.
  \item Download \verb|ps6_progress.zip|.

    Write a function \verb|longest_increasing_subsequence(scores)|
    which takes a list of scores, and returns the longest (strictly)
    increasing subsequence of those scores. Write it in a bottom-up
    manner (because the Python recursion stack is limited).

    Note that, assuming your subproblems from part (a) only find the
    size of the best result, you should also keep parent pointers so
    that you can reconstruct the actual subsequence.

    Submit \verb|progress.py| to the class website.

  \end{enumerate}

\item {\bf (20 points)} Image Resizing

  In a recent paper, ``Seam Carving for Content-Aware Image
  Resizing'', Shai Avidan and Ariel Shamir describe a novel method of
  resizing images. You are welcome to read the paper, but we recommend
  starting with the YouTube video:

  \centerline{\url{http://www.youtube.com/watch?v=vIFCV2spKtg}}

  \noindent
  Both are linked to
  from the Problem Sets page on the class website. After you've
  watched the video, the terminology in the rest of this problem will
  make sense.

  If you were paying attention around time 1:50 of the video, then you
  can probably guess what you're going to have to do. You are given an
  image, and your task is to calculate the best vertical seam to
  remove. A \emph{vertical seam} is a connected path of pixels, one
  pixel in each row. We call two pixels \emph{connected} if they are
  vertically or diagonally adjacent. The \emph{best} vertical seam is
  the one that minimizes the total ``energy'' of pixels in the seam.

  For some reason, the video didn't spend much time on the most
  interesting part---dynamic programming---so here's the algorithm:

  \begin{description}
  \item[Subproblems:] For each pixel $(i,j)$, what is the lower-energy
    seam that starts at the top row of the image, but ends at $(i,j)$?
  \item[Relation:] Let \verb|dp[i,j]| be the solution to subproblem
  $(i,j)$. Then \\
  \verb|dp[i,j] = min(dp[i,j-1],dp[i-1,j-1],dp[i+1,j-1]) + energy(i,j)|
  \item[Analysis:] Solving each subproblem takes $O(1)$ time: there
    are three smaller subproblems to look up, and one call to
    \verb|energy()|, which all take $O(1)$ time. There is one
    subproblem for each pixel, so the running time is $\Theta(A)$,
    where $A$ is the number of pixels, i.e., the area of the image.
  \end{description}

  Download \verb|ps6_image.py|. You will also need installed PIL
  (Python Imaging Library, freely available from
  \url{http://www.pythonware.com/products/pil/}), and Tkinter if you
  want to view images.  If you are using Athena (Linux),
  \verb|add -f 6.006| and run \verb|python2.5|.

  In \verb|ResizeableImage.py|, write a function
  \verb|best_seam(self)| that returns a list of coordinates
  corresponding to the cheapest vertical seam to remove, e.g., \\
  $[(5,0), (5,1), (4,2), (5,3), (6,4)]$.  You should implement the
  dynamic program described above in a bottom-up manner.

  \verb|ResizeableImage| inherits from \verb|ImageMatrix|.  You should
  use the following components of \verb|ImageMatrix| in your dynamic
  program:
  \begin{itemize}
  \item \verb|self.energy(i,j)| returns the energy of a pixel. This
  takes $O(1)$ time, but the constant factor is sizeable.
  \item \verb|self.width| and \verb|self.height| are the width and height
        of the image, respectively.
  \end{itemize}

  Test your code using \verb|test_image.py|, and submit
  \verb|ResizeableImage.py| to the class website. You can also view
  your code in action by running \verb|gui.py|. Included with the
  problem set are two differently sized versions of the same sunset
  image. If you remove enough seams from the sunset image, it should
  center the sun.

  Also, please try out your own pictures (most file formats should
  work), and send us any interesting before/after shots.

\end{enumerate}

\end{document}
